\documentclass{ctexart}

\usepackage{fancyhdr}
\usepackage{listings}
\usepackage{xcolor}
\usepackage{amsmath}

% Codes settings
\lstset{
	language=Python,
	numbers = left,
	numberstyle = \tiny,
	basicstyle = \small,
	keywordstyle = \color{blue!70}\bfseries,
	stringstyle = \color{red},
	commentstyle =\color[rgb]{0.6,.6,.6}\itshape,
	identifierstyle={},
	frame = shadowbox,
	rulesepcolor= \color{ red!20!green!20!blue!20},
	escapeinside=``
}
\pagestyle{fancy}
\rfoot{\thepage}
\cfoot{}
\lfoot{\itshape Jeffreyyao@pku.edu.cn}

\title{计算物理大作业}
\author{姚铭星}
\date{1700011321}
 		
\begin{document}
\maketitle
\thispagestyle{fancy}
\section{第一题}
\subsection{思路}
\section{第二题:高斯求积}
\subsection{思路}
具有如下形式的积分问题可以用高斯求积来解决:
$$\int_{a}^{b} w(x)f(x)dx$$
解决高斯求积问题的关键是找到一组在该权函数下的正交基。
在定义内积
\[ (f,g)=\int_{a}^{b} w(x)f(x)g(x)dx \]
的前提下,权重以及正交基满足下列关系:
\begin{align*}
p_{-1}(x)&=0\\
p_0 (x)&=1\\
p_{j+1}(x)&=(x-a_j)p_{j}(x)-b_jp_{j-1}(x)
\end{align*}
其中:
\begin{align*}
a_j&=\frac{(xp_j,p_j)}{(p_j,p_j)}\\
b_j&=\frac{(p_{j},p_{j})}{(p_{j-1},p_{j-1})}
\end{align*}
设$x_{\mu}$是正交多项式$p_N(x)$的第$\mu$个根,则系数满足:
\[w_j=\frac{(p_{N-1}(x),p_{N-1}(x))}{p_{N-1}(x_j)p'_{N}(x_j)}\]

\end{document}