\documentclass[a4paper]{ctexart}

\usepackage{fancyhdr}
\usepackage{listings}
\usepackage{xcolor}
\usepackage{amsmath}
\usepackage{algorithm}
\usepackage{graphicx}
\usepackage[noend]{algpseudocode}
\usepackage{physics}
\usepackage{multirow}

% Codes settings
\lstset{
	language=Python,
	numbers = left,
	numberstyle = \tiny,
	basicstyle = \small,
	keywordstyle = \color{blue!70}\bfseries,
	stringstyle = \color{red},
	commentstyle =\color[rgb]{0.6,.6,.6}\itshape,
	identifierstyle={},
	frame = shadowbox,
	rulesepcolor= \color{ red!20!green!20!blue!20},
	escapeinside=``
}

%Page setting
\pagestyle{fancy}
\rfoot{\thepage}
\cfoot{}
\lfoot{\itshape Jeffreyyao@pku.edu.cn}
\setlength{\abovecaptionskip}{0.05cm}
\setlength{\belowcaptionskip}{0.05cm}

\title{计算物理第三次开卷考试}
\author{姚铭星 1700011321}
\date{}

\begin{document}
\maketitle
\section{第一题:原子的激发谱}
\subsection{基态的求解}
对于数值求解不含时薛定谔方程:
\begin{equation}
\left[-\frac{1}{2} \frac{d^{2}}{d x^{2}}+V(x)\right] \psi(x)=E_{n} \psi(x)
\end{equation}
可以将二阶导数该为有限差分
\begin{equation}
\partial_{x}^{2} \psi\left(x_{i}\right)=\frac{\psi\left(x_{i+1}\right)-2 \psi\left(x_{i}\right)+\psi\left(x_{i-1}\right)}{(\Delta x)^{2}}+o\left((\Delta x)^{2}\right)
\end{equation}
这样可以将哈密顿量表示为三对角对称矩阵的形式:
\begin{equation}
\left(H_{0}\right)_{i j}=\left\{\begin{array}{ll}{\frac{1}{(\Delta x)^{2}}-\frac{1}{\sqrt{x_{i}^{2}+2}},} & {i=j} \\ {\frac{-1}{2(\Delta x)^{2}},} & {|i-j|=1} \\ {0,} & {|i-j|>1}\end{array}\right.
\end{equation}
注意,此处已经使用了边界条件:
\begin{equation}
\psi(x_{-1})=\psi(x_{N})=0
\end{equation}
这样问题变为求解本征值问题:
\begin{equation}
H_0\psi = E_0 \psi
\end{equation}
由于已经给定近似的能量值$E_0=-0.48$,利用原点位移的方法可以直接将问题转化为求解差值的问题。这样可以保证反幂法的顺利运行。

定义新的矩阵$A$为:
\begin{equation}
A = H_0 + 0.48I
\end{equation}
此时利用Cholesky分解(在第一次作业中已经实现过)实现类似于求逆的操作

\subsection{时间演化}
对于加上电场之后的含时的薛定谔方程,与上问中的哈密顿量只差相差一个对角阵:
\begin{equation}
\mathcal{H}(t)=\mathcal{H}_{0}+x E(t)
\end{equation}
因此此时哈密顿量仍然保有三对角形式,并且为实对称矩阵。在时空上划定格点将方程离散化后得到:

\begin{equation}
\begin{aligned}
\psi_{i}^{(j)}&=\psi\left(x_{i}, t_{j}\right)\\
i\frac{\psi_{i}^{(j+1)}-\psi_{i}^{(j)}}{\Delta t}&=\mathcal{H}(t)\psi(t)
\end{aligned}
\end{equation}

为了保证体系的守恒量不受破坏,我们采用辛算法Crank-Nicolson(欧拉中点算法),其格式为:
\begin{equation}
\mathrm{i} \frac{\psi_{i}^{(j+1)}-\psi_{i}^{(j)}}{\Delta t}=\sum_{l} H\left(t_{j+1 / 2}\right)_{i l} \psi_{l}\left(t_{j+1 / 2}\right)=\sum_{l} \frac{1}{2} H\left(t_{j+1 / 2}\right)_{i l}\left(\psi_{l}^{(j)}+\psi_{l}^{(j+1)}\right)
\end{equation}
这样,每一步递推变成求解线性方程问题:
\begin{equation}
\left(1+\frac{\mathrm{i} \Delta t}{2} H\left(t_{j+1 / 2}\right)\right) \psi^{(j+1)}=\left(1-\frac{\mathrm{i} \Delta t}{2} H\left(t_{j+1 / 2}\right)\right) \psi^{(j)}
\end{equation}
而不难发现,此时的Crank-Nicolson形式有幺正性质,即满足:
\begin{equation}
U_jU^{\dagger}_j=1
\end{equation}

\end{document}